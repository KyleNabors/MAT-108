\item Let $x\in\bR$ be positive, and consider the proposition:
\textit{If $x$ is irrational, then $\sqrt{x}$ is irrational.}
\begin{enumerate}
\item Complete the following proof by contraposition: \textit{Assume
$\sqrt{x}$ is rational. Then there exist $m,n\in\bZ$ such that $\sqrt{x} = \frac{m}{n}$ with $m$ and $n$ being coprime integers and $n > 0$.

Now, let's square both sides of the equation to eliminate the square root:
$x = \left(\frac{m}{n}\right)^2 = \frac{m^2}{n^2}$.

To show that $x$ is rational, we need to prove that it can be expressed as a ratio of two integers. From the equation above, we see that $x$ is expressed as the ratio of $m^2$ and $n^2$.

We know that both $m$ and $n$ are integers, as given in the problem statement. Since the product of two integers is also an integer, $m^2$ and $n^2$ are both integers (e.g., if $m$ and $n$ are integers, then $m^2 = m \cdot m$ and $n^2 = n \cdot n$, and the products are also integers).

Since $x = \frac{m^2}{n^2}$ and both $m^2$ and $n^2$ are integers, we can conclude that $x$ is rational. This is because the definition of a rational number is any number that can be expressed as the ratio of two integers, which is the case for $x$. Therefore, by contraposition, if $x$ is irrational, then $\sqrt{x}$ must be irrational as well.
\end{enumerate}