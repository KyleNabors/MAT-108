\documentclass{article}
\usepackage[utf8]{inputenc}
\usepackage{amssymb,amsmath,titling,enumitem}
\usepackage[a4paper, margin=0.75in]{geometry}
\setlength{\droptitle}{-30pt}

\title{Homework 4}
\author{MAT 108}
\date{Due 11:59pm, 5/5/2023\\ 
\vspace{0.1cm}
Submit to Gradescope}

\newcommand{\bR}{\mathbb{R}}
\newcommand{\bQ}{\mathbb{Q}}
\newcommand{\bC}{\mathbb{C}}
\newcommand{\bZ}{\mathbb{Z}}
\newcommand{\bN}{\mathbb{N}}

\begin{document}

\maketitle

{\large

\begin{enumerate}[labelindent=0pt,leftmargin=0pt]

    \setlength{\itemsep}{13pt} 

    \item Find all of the inductive subsets of $\bN$ that do not contain any powers of $2$. Justify that your list of sets is complete.

    Let $S$ be an inductive subset of $\bN$ that does not contain any powers of $2$. 
    Since $S$ contains $1$, it cannot contain any power of $2$ greater than $1$. 
    Therefore, $S$ can contain at most one power of $2$.
    
    Given that $S$ contains $1$ but no power of $2$, then it must contain all positive odd integers.
    Since any odd integer $n$ can be written as $n=2k+1$ for some integer $k$, and $S$ contains $1$ and $2k$, it must also contain $n$. 
    Therefore, the set $S_1={1,3,5,\ldots}$ is an inductive subset of $\bN$ that does not contain any powers of $2$.
    
    Given that $S$ contains $1$ and $2^k$ for some positive integer $k$, then $S$ cannot contain any other power of $2$, since $2^{k+1}$ is not in $S$. 
    It must hold that, $S$ contains all positive integers less than $2^k$. Otherwise there would exist a smallest positive integer $n$ not in $S$, and then $n-1$ would be in $S$ but not $n$, contradicting the fact that $S$ is inductive. 
    Therefore, $S$ is of the form ${1,2,3,\ldots,2^k-1}$, and $S$ is an inductive subset of $\bN$ that does not contain any other powers of $2$.
    
    Thus, the complete list of inductive subsets of $\bN$ that do not contain any powers of $2$ is ${1,3,5,\ldots}$ and ${1}$ together with the sets of the form ${1,2,3,\ldots,2^k-1}$ for $k\geq 1$.




    \item Suppose $A,B\subseteq\bN$ are inductive. Are the following conclusions valid? Give either a proof or a counterexample.

    \begin{enumerate}
    \item $A\cup B$ is inductive.
    
    Suppose $A$ and $B$ are inductive subsets of $\mathbb{N}$. To show that $A\cup B$ is inductive, we need to verify that $1\in A\cup B$ and that whenever $n\in A\cup B$, then $n+1\in A\cup B$. Since $A$ and $B$ are both inductive, we know that $1\in A$ and $1\in B$, so $1\in A\cup B$. Now suppose $n\in A\cup B$. If $n\in A$, then $n+1\in A$ since $A$ is inductive, and therefore $n+1\in A\cup B$. If instead $n\in B$, then $n+1\in B$ since $B$ is inductive, and therefore $n+1\in A\cup B$. In either case, we see that $A\cup B$ is inductive. Therefore, the conclusion is valid.
    

    \item $A\cap B$ is inductive.
    
    Consider the inductive subsets $A={1,3,5,\ldots}$ and $B={2,4,6,\ldots}$. Then $A\cap B=\emptyset$, which is not inductive since it does not contain $1$. Therefore, the conclusion is not valid.


    \item $A-B$ is inductive.
   
    Suppose $A$ and $B$ are inductive subsets of $\mathbb{N}$. To show that $A-B$ is inductive, we need to verify that $1\in A-B$ and that whenever $n\in A-B$, then $n+1\in A-B$. Since $1\in A$ and $1\notin B$, we have $1\in A-B$. Now suppose $n\in A-B$. Then $n\in A$ and $n\notin B$. Since $A$ is inductive, we have $n+1\in A$, and since $n\notin B$, we have $n+1\notin B$. Therefore, $n+1\in A-B$, and we see that $A-B$ is inductive. Therefore, the conclusion is valid.
    

    \item $A-B$ is not inductive.
   
    Consider the inductive subsets $A={1,3,5,\ldots}$ and $B={2,4,6,\ldots}$. Then $A-B={1,3,5,\ldots}$, which is inductive. Therefore, the conclusion is not valid.


    \end{enumerate}



    
    \item Use (generalized) induction to prove that $$\prod_{k=2}^n\frac{k^2-1}{k^2}=\frac{n+1}{2n}$$ for all $n\geq 2$. (Here $\Pi$ denotes a product just as $\Sigma$ denotes a sum.)


    Proof: Let $S=\{n:\prod_{k=2}^n\frac{k^2-1}{k^2}=\frac{n+1}{2n}\}$ We will use induction to show $S \in \mathbb{N}$ for all $n\geq2$
    
    Base case: n=2, $\prod_{k=2}^n\frac{2^2-1}{2^2}=\frac{2+1}{2*2}$, so the claim holds for n=2
    
    Inductive Step: Let $k \geq 2$ be given and assume $\prod_{k=2}^n\frac{k^2-1}{k^2}=\frac{n+1}{2n}$ holds for $n=k$.
    

    Thus $\prod_{k=2}^{k+1}\frac{k^2-1}{k^2}=(\prod_{k=2}^k\frac{k^2-1}{k^2}) * \frac{(k+1)^2-1}{(k+1)^2}$ 
   
    $=(\frac{n+1}{2n}) * \frac{(k+1)^2-1}{(k+1)^2}$ (By induction hypothesis)

    $=\frac{k+1}{2k} * \frac{(k+1)^2-1}{(k+1)^2}$ (By assumption) 

    $=\frac{k+2}{2(k+1)}$ $=\frac{(k+1)+1}{2(k+1)}$ (By algebra)

    Thus $\prod_{k=2}^n\frac{k^2-1}{k^2}=\frac{n+1}{2n}$ holds for $n=k+1$ and the proof of the induction step is complete. 

    By induction the statement $\prod_{k=2}^n\frac{k^2-1}{k^2}=\frac{n+1}{2n}$ holds for all $n\geq 2$
    




    \item Use strong induction to prove that every natural number greater than $1$ is equal to a product of prime numbers. (You do not need to prove any uniqueness claims about this product.)




    \item Find two relations $R$ and $S$ on the set $\{1,2,3\}$ that satisfy $|R|,|S|\geq 3$ and $R\circ S=\emptyset$.




    \item Let $n\geq 2$, and let $A=\{1,2,...,2n\}$. Find a relation $R$ on $A$ such that $R\circ R$ is the identity relation on $A$, but $(k,k)\not\in R$ for all $k\in A$.

\end{enumerate}

}
\end{document}
