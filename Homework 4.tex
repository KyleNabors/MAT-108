\documentclass{article}
\usepackage[utf8]{inputenc}
\usepackage{amssymb,amsmath,titling,enumitem}
\usepackage[a4paper, margin=0.75in]{geometry}
\setlength{\droptitle}{-30pt}

\title{Homework 4}
\author{MAT 108}
\date{Due 11:59pm, 5/5/2023\\ 
\vspace{0.1cm}
Submit to Gradescope}

\newcommand{\bR}{\mathbb{R}}
\newcommand{\bQ}{\mathbb{Q}}
\newcommand{\bC}{\mathbb{C}}
\newcommand{\bZ}{\mathbb{Z}}
\newcommand{\bN}{\mathbb{N}}

\begin{document}

\maketitle

{\large

\begin{enumerate}[labelindent=0pt,leftmargin=0pt]

    \setlength{\itemsep}{13pt} 

    \item Find all of the inductive subsets of $\bN$ that do not contain any powers of $2$. Justify that your list of sets is complete.
    
    Let $A\subseteq\bN$ subject to $2^k \notin A$ for all $k\in\bN$. Since A is inductive, then for all $n\in A$ then $n+1\in A$. 
    Considering the following, let $2^k-1\in A$ for some $k\in\bN$. Because A is inductive it implies that $(2^k-1)+1\in A$. 
    However $(2^k-1)+1=2^k\notin A$ by assumption, therefore $2^k \neq 2^k$. This implies that for A to be an inductive set of $\bN$ that does not contain any powers of $2$ it must be that $A=\emptyset$




    \item Suppose $A,B\subseteq\bN$ are inductive. Are the following conclusions valid? Give either a proof or a counterexample.

    \begin{enumerate}
    \item $A\cup B$ is inductive.
    
    Let $n \in A \cup B$. Consider the following two cases 

    Case 1: $n \in A$
    Since $A$ is inductive, we know that $n+1 \in A$. Thus, $n+1 \in A \cup B$.

    Case 2: $n \in B$
    Since $B$ is inductive, we know that $n+1 \in B$. Thus, $n+1 \in A \cup B$.

    In both cases, if $n \in A \cup B$, then $n+1 \in A \cup B$. Therefore, $A \cup B$ is inductive.


    \item $A\cap B$ is inductive.
    
    Let $A={1,3,5,\ldots}$ and $B={2,4,6,\ldots}$. 
    Then $A\cap B=\emptyset$, which is not inductive since it does not contain $1$. 
    Therefore, $A\cap B$ is not intductive. 


    \item $A-B$ is inductive.
   
    Let $A = \bN$ and $B = {1}$. Both $A$ and $B$ are inductive subsets of $\mathbb{N}$. 
    
    Therefore, $A - B = \bN - \{1\} = \{2,3,4,\ldots\}$:

    In this case, $A - B$ does not contain $1$. Therefore, $A - B$ is not inductive.


    \item $A-B$ is not inductive.
   
    Let $A={1,3,5,\ldots}$ and $B={2,4,6,\ldots}$. Both $A$ and $B$ are inductive subsets of $\mathbb{N}$. 
    
    Then $A-B={1,3,5,\ldots}$, which is inductive. 


    \end{enumerate}


    \item Use (generalized) induction to prove that $$\prod_{k=2}^n\frac{k^2-1}{k^2}=\frac{n+1}{2n}$$ for all $n\geq 2$. (Here $\Pi$ denotes a product just as $\Sigma$ denotes a sum.)


    Proof: We will prove by induction that for all $n\geq 2$, $\prod_{k=2}^n\frac{k^2-1}{k^2}=\frac{n+1}{2n}$
    
    Base case: n=2, $\prod_{k=2}^n\frac{2^2-1}{2^2}=\frac{2+1}{2*2}$, so the claim holds for n=2
    
    Inductive Step: Let $k \geq 2$ be given and assume $\prod_{k=2}^n\frac{k^2-1}{k^2}=\frac{n+1}{2n}$ holds for $n=k$.
    

    Thus $\prod_{k=2}^{k+1}\frac{k^2-1}{k^2}=(\prod_{k=2}^k\frac{k^2-1}{k^2}) * \frac{(k+1)^2-1}{(k+1)^2}$ 
   
    $=(\frac{n+1}{2n}) * \frac{(k+1)^2-1}{(k+1)^2}$ (By induction hypothesis)

    $=\frac{k+1}{2k} * \frac{(k+1)^2-1}{(k+1)^2}$ (By assumption) 

    $=\frac{k+2}{2(k+1)}$ $=\frac{(k+1)+1}{2(k+1)}$ (By algebra)

    Thus $\prod_{k=2}^n\frac{k^2-1}{k^2}=\frac{n+1}{2n}$ holds for $n=k+1$ and the proof of the induction step is complete. 

    By induction the statement $\prod_{k=2}^n\frac{k^2-1}{k^2}=\frac{n+1}{2n}$ holds for all $n\geq 2$
    
  


    \item Use strong induction to prove that every natural number greater than $1$ is equal to a product of prime numbers. (You do not need to prove any uniqueness claims about this product.)

    Proof: We will prove by induction that for all $1 < n \in \bN$ is equal to the product of two prime numbers. 

    Base case: $n=2$, Since 2 is a prime number it can be represented as a product of the prime number 1 and itself. 

    Assume that for all natural numbers $m$ with $2 \leq m \leq n$, $m$ can be represented as a product of prime numbers.
 
    We want to show that $(n+1)$ can be represented as a product of prime numbers. We will consider the following two cases. 

    Case 1:  $(n+1)$ is prime

    If $(n+1)$ is prime, then it is itself a product of prime numbers this being the prime number one and itself.

    Case 2: $(n+1)$ is composite
    
    If $(n+1)$ is composite, then there exists $a,b \in\bN$ with $1 \leq a < n+1$ and $1 \leq b < n+1$ such that $(n+1) = a \cdot b$. 
    By inductive hypothesis, we know that $a$ and $b$ can be represented as a product of prime numbers. 
    Therefore, $(n+1)$ can also be represented as a product of prime numbers, as it is the product of $a$ and $b$.

    In both cases, $(n+1)$ can be represented as a product of prime numbers. By strong induction, every natural number greater than $1$ is equal to a product of prime numbers.


    \item Find two relations $R$ and $S$ on the set $\{1,2,3\}$ that satisfy $|R|,|S|\geq 3$ and $R\circ S=\emptyset$.

    $R = \{(1,1),(2,1),(3,1 )\} $ and $S=\{(2,1),(2,2),(2,3)\}$

    Since there is no pair $(a, c)$ for which there exists a $b$ such that $(a, b) \in R$ and $(b, c) \in S$, we have $R \circ S = \emptyset$. Thus, the given relations $R$ and $S$ satisfy the required conditions.

    \item Let $n\geq 2$, and let $A=\{1,2,...,2n\}$. Find a relation $R$ on $A$ such that $R\circ R$ is the identity relation on $A$, but $(k,k)\not\in R$ for all $k\in A$.

    Given that $R$ is a relation on A such that $R\circ R$ is the identity relation on $A$ but $(k,k)\not\in R$ for all $k\in A$, this implies that $R$ is not the identity relation itself, i.e $R=\{(1,1),(2,2),...,(2n,2n)$.

    Thus, we can pair each number $k\in \{1,2,...,n\}$ with a counterpart in the set ${n+1, n+2,...,2n}$.
    
    Thus, we define the relation $R$ as follows:
   
    $R=\{(k,k+n):k\in\{1,2,...,n\}\}\lor\{(k+n,k):k\in\{1,2,...,n\}\}$

\end{enumerate}

}
\end{document}
