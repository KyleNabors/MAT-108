\documentclass{article}
\usepackage[utf8]{inputenc}
\usepackage{amssymb,amsmath,titling,enumitem}
\usepackage[a4paper, margin=0.75in]{geometry}
\setlength{\droptitle}{-30pt}
\title{Homework 1 Kyle Nabors Submission}
\author{MAT 108}
\date{Due 11:59pm, 4/7/2023\\
\vspace{0.1cm}
Submit to Gradescope}
\newcommand{\bR}{\mathbb{R}}
\newcommand{\bQ}{\mathbb{Q}}
\newcommand{\bC}{\mathbb{C}}
\newcommand{\cB}{\mathcal{B}}
\newcommand{\cC}{\mathcal{C}}
\newcommand{\bZ}{\mathbb{Z}}
\newcommand{\bN}{\mathbb{N}}
\newcommand{\cG}{\mathcal{G}}
\newcommand{\cH}{\mathcal{H}}
\begin{document}
\maketitle
\begin{enumerate}[labelindent=0pt,leftmargin=0pt]
{\large}
    \setlength{\itemsep}{13pt}
    \item Let $P$ and $Q$ be propositions. Make a truth table for each of
the following, and label it as a tautology, a contradiction, or neither.
    \begin{enumerate}
        \item $(P\,\land\sim\! P)\lor P$
        
        \begin{flushleft}
         \begin{tabular}{||c c c c||} 
        \hline
        P & $\sim\! P$ & $(P\,\land\sim\! P)$ & $(P\,\land\sim\! P)\lor P$  \\ [0.5ex] 
        \hline\hline
        T & F & F & T \\
        F & T & F & F \\
        \hline
        \end{tabular}
        \end{flushleft}
        
        \item $(P\land Q)\Rightarrow P$
        
        \begin{flushleft}
         \begin{tabular}{||c c c c||} 
        \hline
        P & Q & $(P\land Q)$ & $(P\land Q)\Rightarrow P$  \\ [0.5ex] 
        \hline\hline
        T & T & T & T \\
        T & F & F & T \\
        F & T & F & T \\
        F & F & T & F \\
        \hline
        \end{tabular}
        \end{flushleft}
        
        \item $(P\lor Q)\Rightarrow P$
                
        \begin{flushleft}
         \begin{tabular}{||c c c c||} 
        \hline
        P & Q & $(P\lor Q)$ & $(P\lor Q)\Rightarrow P$  \\ [0.5ex] 
        \hline\hline
        T & T & T & T \\
        T & F & T & T \\
        F & T & T & F \\
        F & F & F & T \\
        \hline
        \end{tabular}
        \end{flushleft}
        
        \item $(P\,\lor\sim\! P)\Rightarrow (Q\,\land\sim\!Q)$
                  
        \begin{flushleft}
         \begin{tabular}{||c c c c c||} 
        \hline
        P & Q & $(P\,\lor\sim\! P)$ & $(Q\,\land\sim\!Q)$ & $(P\,\lor\sim\! P)\Rightarrow (Q\,\land\sim\!Q)$ \\ [0.5ex] 
        \hline\hline
        T & T & T & F & F \\
        T & F & T & F & F \\
        F & T & T & F & F \\
        F & F & F & F & T \\
        \hline
        \end{tabular}
        \end{flushleft}
        
        \item $(\sim\! P\land (\sim Q\Rightarrow P))\land (Q\Rightarrow P)$

        \begin{flushleft}
        \begin{tabular}{||c c c c c c ||} 
        \hline
        P & Q & $(\sim Q\Rightarrow P)$ & $(\sim\! P\land (\sim Q\Rightarrow P))$ & $(Q\Rightarrow P)$ & $(\sim\! P\land (\sim Q\Rightarrow P))\land (Q\Rightarrow P)$ \\ [0.5ex] 
        \hline\hline
        T & T & T & F & T & F \\
        T & F & T & F & T & F \\
        F & T & T & T & F & F \\
        F & F & F & F & T & F \\
        \hline
        \end{tabular}
        \end{flushleft}
        
        \item $(P\,\land \sim\! Q)\Rightarrow\,\sim\!(P\Leftrightarrow Q)$

        \begin{flushleft}
        \begin{tabular}{||c c c c c||} 
        \hline
        P & Q & $(P\,\land \sim\! Q)$ & $\sim\!(P\Leftrightarrow Q)$ & $(P\,\land \sim\! Q)\Rightarrow\,\sim\!(P\Leftrightarrow Q)$ \\ [0.5ex] 
        \hline\hline
        T & T & F & F & T \\
        T & F & T & T & T \\
        F & T & F & T & T \\
        F & F & F & F & T \\
        \hline
        \end{tabular}
        \end{flushleft}
    \end{enumerate}

\vspace{25pt}

    \item \begin{enumerate} \item Construct a proposition from $P$ and $Q$
using only $\lor$ and $\sim$ that has the following truth table.
(Parentheses are allowed, and you can use symbols multiple times.) You do
not need to justify your answer.

    \begin{center}
    \begin{flushtable}
    \renewcommand{\arraystretch}{1.2}
    \centering
    \begin{tabular}{|l|l|l|ll}
    \hline
    $P$ & $Q$ & ? \\ \hline
    T & T & F\\
    T & F & F\\
    F & T & T\\
    F & F & F\\ \hline
    
    \end{tabular}
    \end{flushtable}
    \end{center}

    \subitem $\sim\!P \land Q$
    \vspace{5pt}
    
\item Is it possible to obtain the truth table in part (a) using only
$\Rightarrow$ and $\Leftrightarrow$ instead? If so, do it. If not, say why
it cannot be done.
\end{enumerate}

No it is not possible to construct a proposition that satisfies the truth table given in a. This is because given the tools provided there is no way to make a false statement given that both operators are true. Therefore we cannot satisfy the first row of the table and the entire problem is impossible. 


\item Paul, Quinn, and Rae are deciding whether to go out this evening
or just stay home. Quinn will go out if and only if both Paul and Rae do,
Paul will go out if Rae stays home, and Rae will go out only if exactly
one other person goes with her. What happens? Justify your answer.

\subitem Let Q be if Quinn is going out, let R be if Rae is going out, let P be if Paul is going out.

\subitem Q(x) $\Leftrightarrow P \land Q$
\subitem P(x) if $\sim\!R$
\subitem R(x)  if $(P, \sim\!Q)\lor(\sim\!R)$

\subitem R is true if x = $(P,Q,R) \lor (\sim\!P,\sim\!Q,R) \lor (\sim\!P,\sim\!Q) \lor (P,\sim\!Q,\sim\!R)$
\subitem P is true if x = $(P,R,Q) \lor (P,Q, \sim\!R) \lor (\sim\!P,Q,R) \lor (P,\sim\!,Q,\sim\!R) \lor (P, \sim\!Q, R \lor (\sim\!P,\sim\!Q,R)$
\subitem Q is true if x = $(P,\sim\!Q,R) \lor (\sim\!P,Q,R) \lor (P,Q,\sim\!R) \lor (\sim\!P,Q,\sim\!R) \lor (P,\sim\!Q,\sim\!R) \lor (\sim\!P,\sim\!Q,R)$

\subitem $(P,\sim\!Q,\sim\!R)$ is common between all three elements.
\subitem This means Paul will go out, Rae will stay in, and Quinn will stay in. 

\end{enumerate}
\end{document}