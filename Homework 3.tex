\documentclass{article}
\usepackage[utf8]{inputenc}
\usepackage{amssymb,amsmath,titling,enumitem}
\usepackage[a4paper, margin=0.5in]{geometry}
\setlength{\droptitle}{-30pt}
\title{Homework 3}
\author{MAT 108}
\date{Due 11:59pm, 4/21/2023\\
\vspace{0.1cm}
Submit to Gradescope}
\newcommand{\bR}{\mathbb{R}}
\newcommand{\bQ}{\mathbb{Q}}
\newcommand{\bC}{\mathbb{C}}
\newcommand{\bZ}{\mathbb{Z}}
\newcommand{\bN}{\mathbb{N}}
\begin{document}
\maketitle
{\large
\begin{enumerate}[labelindent=0pt,leftmargin=0pt]
\setlength{\itemsep}{13pt}
\item\begin{enumerate}
\item Make a truth table for $((P\land Q)\lor R)\Leftrightarrow
((P\lor R)\land (Q\lor R))$.

\begin{table}[hbt!]
    \centering
    \begin{tabular}{c|c|c|c|c|c|c|c|c}
        P & Q & R & $P \land Q$ & $(P \land Q) \lor R$ & $P \lor R$ & $Q \lor R$ & $(P \lor R) \land (Q \lor R)$ & $((P \land Q) \lor R) \Leftrightarrow (P \lor R) \land (Q \lor R)$ \\
        T & T & T & T & T & T & T & T & T \\
        T & T & F & T & T & T & T & T & T \\
        T & F & T & F & T & T & T & T & T \\
        T & F & F & F & F & T & F & F & T \\
        F & T & T & F & T & T & T & T & T \\
        F & T & F & F & F & F & T & F & T \\
        F & F & T & F & T & T & T & T & T \\
        F & F & F & F & F & F & F & F & T \\
    \end{tabular}
    \caption{Caption}
    \label{tab:my_label}
\end{table}

\item Use part (a) to prove that $((P\lor Q)\Rightarrow
R)\Leftrightarrow ((P\Rightarrow R)\land(Q\Rightarrow R))$ is a tautology
without making another truth table.






\item Which of our proof methods does part (b) justify? How so? In
your answer, note why it is relevant that the main connective is ``if and
only if."
\end{enumerate}
\item Let $x\in\bR$ be positive, and consider the proposition:
\textit{If $x$ is irrational, then $\sqrt{x}$ is irrational.}
\begin{enumerate}
\item Complete the following proof by contraposition: \textit{Assume
$\sqrt{x}$ is rational. Then there exist $m,n\in\bZ$ such that ...... Thus
$x$ is rational.}

Lemma: For all $n\in\bR$ if $n \neq 0$ there there exists $a,k\in\bR$ such that $a\geq 0$, k is odd, and $n=x^a k$. 

Proof: Assume $\sqrt{x}$ is rational. 
Then there exist $m,n\in\bR$ with $n \neq 0$ such that $\sqrt{x} = m/n$
By the lemma, $m=x^a j$ and $n=2^b k$ for some $a,b,j,k\in\bR$ with $a,b\geq0$, and  j and k are odd.
Since $\sqrt{x} = m/n$ we have $x = (x^a j / x^b k)^2 = x^2a j^2 / x^2b k^2$,
which in turn shows that $x^(2b+1) k^2 = x^2a j^2$ Since 


\item Contrapose the implications in your proof from part (a) and
reverse their order to give a direct proof of the proposition. (Which
version of the proof is cleaner? (That is rhetorical.))




\end{enumerate}
\item In our class notes there is a proof that $\sqrt{2}\not\in\bQ$.
Use this fact in a proof by contradiction to show that there do not exist
nonzero rational numbers $a$, $b$, and $c$ such that
$a\sqrt{2}+b\sqrt{3}=c$. (We did not prove that $\sqrt{3}$ or $\sqrt{6}$
is irrational, so do not use that.)


Proof: By contradiction assume there does not exist nonzero rational numbers a, b, and c such that $a\sqrt{2} + b\sqrt{3} = c$.
We can rewrite as $a\sqrt{2} =c - b\sqrt{3}$.
Multiplying both sides by $sqrt{2}$ we can rewrite as 
$2a =c\sqrt{2} - b\sqrt{6}$.
We know$2a$ is rational as a is rational by our assumptions. 
Given that $\sqrt{2}$ is irrational implies $c\sqrt{2}$ is irrational which implies 
$c\sqrt{2} - b\sqrt{6}$ is irrational given an irrational number minus a irrational number such as $b\sqrt{6}$ is irrational. 
This implies that $a\sqrt{2} + b\sqrt{3} = c$. is a false statement and is a contradiction. 
Thus our initial assumption that there exist nonzero rational numbers $a$, $b$, and $c$ such that $a\sqrt{2} + b\sqrt{3} = c$ must be false, and there do not exist such rational numbers $a$, $b$, and $c$.







\item Give an example, if possible, of sets $A$, $B$, and $C$ that
satisfy the following. If no example exists, write ``Not possible." You do
not need to include explanations with your answers.
\begin{enumerate}
\item $A\subseteq B$, $B\not\subseteq C$, and $A\subseteq C$.
\subitem $A = {1}, B = {1, 2}, C = {1, 3}$
\item $A\not\subseteq B$, $B\subseteq C$, and $C\subseteq A$.
\subitem Not Possible
\item $A\not\subseteq B$, $B\not\subseteq C$, and $A\subseteq C$.
\subitem $A = {1}, B = {2}, C = {1, 3}$
\item $A\subseteq B$, $B\subseteq C$, and $C\subseteq A$.
\subitem $A = {1}, B = {1}, C = {1}$
\end{enumerate}
\end{enumerate}
}
\end{document}