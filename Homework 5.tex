\documentclass{article}
\usepackage[utf8]{inputenc}
\usepackage{amssymb,amsmath,titling,enumitem}
\usepackage[a4paper, margin=0.75in]{geometry}

\setlength{\droptitle}{-30pt}

\title{Homework 5}
\author{MAT 108}
\date{Due 11:59pm, 5/12/2023\\ 
\vspace{0.1cm}
Submit to Gradescope}

\newcommand{\bR}{\mathbb{R}}
\newcommand{\bQ}{\mathbb{Q}}
\newcommand{\bC}{\mathbb{C}}
\newcommand{\bZ}{\mathbb{Z}}
\newcommand{\bN}{\mathbb{N}}

\begin{document}

\maketitle

{\large

\begin{enumerate}[labelindent=0pt,leftmargin=0pt]

    \setlength{\itemsep}{13pt} 

    \item Let $R$ be a relation on a set $A$. Consider the three properties: reflexive, symmetric, and transitive.\begin{enumerate}
    \item Which property is equivalent to $R\subseteq R^{-1}$? Justify your answer.

    The inverse relation $R^{-1}$ is defined as ${(y, x) | (x, y) \in R}$. So, if $R\subseteq R^{-1}$, then for every  $(x, y)$ in $R$, the pair $(y, x)$ must also be in $R$.

    A symmetric relationship is defined as for all $x, y \in A$, if $(x, y) \in R$ then $(y, x) \in R$. Thus is must be that $R\subseteq R^{-1}$ is equivalent to the symmetric property. 
    
    \item Which property is equivalent to $\text{Id}_A\subseteq R$? Justify your answer.

    The identity relation on $A$, denoted by $\text{Id}_A$, is defined as ${(x, x) | x \in A}$. 
    Thus, if $\text{Id}_A\subseteq R$, then every pair of the form $(x, x)$ is in $R$.

    A relation $R$ on a set $A$ is reflexive iff, for all $x \in A$, $(x, x) \in R$. 
    Thus, it must be that $\text{Id}_A\subseteq R$ is reflexive. 
    
    \item Find a definition of the last remaining property that uses only sets rather than elements (similar to how parts (a) and (b) give definitions using only sets and not elements).

    The last of the three remaining properties is transitive. 
    The following is an alternative definition of transitive. 
    A relation $R$ is transitive iff $R \circ R \subseteq R$.
    
    \end{enumerate}
    
    \item Define a relation on $\bR$ that is symmetric and transitive, but not reflexive. Justify your answer.

    $\forall x, y \in \mathbb{R}$, $(x, y) \in R$ iff $x*y \geq 1$.

    This is not reflexive because it is not possible for x and y to both equal zero. 
    It is symmetric because $\forall x, y \in \mathbb{R}$ $xy = yx$
    It is transitive because $(a, b) \in\bR $ and $(b,c) \in\bR $ it is also true that $(a,c) \in\bR $ 

    \item Let $R$ be the relation on $\bN$ defined by $a R b$ iff there exists $n\in\bN$ such that $ab=n^2$. Prove that $R$ is an equivalence relation.

    Proof: 

    Reflexive: When considering $aRa$, if $aRa$ then $\exists n \in\bN$ subject to $a*a = n^2 \Rightarrow a*a = n^2$ 

    Symmetric: If $aRb$ then $\exists\in\bN$ subject to $a*b=n^2$ by commutativity $b*a = n^2$

    Transitive: Consider the following two statements. 
    $aRb \Rightarrow \exists \in \bN$ Subject to $ab=n^2$
    $bRc \Rightarrow \exists \in \bN$ Subject to $bc=m^2$
    This implies $(ab)(bc) = n^2m^2$
    Since $a,b,c \in\bN $ then $ac*b^2 = n^2m^2  \leftrightarrow  ac = (n^2m^2)/b^2  \leftrightarrow  ac = (nm/b)^2 $

    Thus R is an equivalence relation

    \item Recall that for each $n\in\bZ$ we define a relation $R_n$ on $\bZ$ by $a R_n b$ iff $n\,|\,(a-b)$. Find a value for $n$ such that $R_n=R_{12}\circ R_8$. Prove that your answer is correct. (Optional: Can you determine the relationship between $a$, $b$, and $n$ that guarantees $R_n=R_a\circ R_b$?)

    Given $R_{12}$ and $R_8$, $a R_{12}\circ R_8 b$ holds if $\exists c \in \mathbb{Z}$ such that $12|(a-c)$ and $8|(c-b)$. 
    Thus $a$ and $b$ leave the same remainder when divided by the least common multiple of 12 and 8. 
    The LCM of 12 and 8 is 24, so $R_{24}=R_{12}\circ R_8$. 

    Proof: Consider the following two cases 
    Case 1: $R_{24} \subseteq R_{12}\circ R_8$.  
    We can write $a = 24k + r$ and $b = 24l + r$ for $k, l \in\bZ$ and $0 \leq r < 24$. We can choose $c = 24k + r$, then $12|(a-c) = 12(k-l)$ and $8|(c-b) = 8(k-l)$, so $a R_{12}\circ R_8 b$.

    Case 2: $R_{24} \supseteq R_{12}\circ R_8$
    If $a R_{12}\circ R_8 b$, $\exists c$ such that $12|(a-c)$ and $8|(c-b)$. The least common multiple of 12 and 8 is 24, so $24|(a-b)$, hence $a R_{24} b$.

    Therefore, $R_{24}=R_{12}\circ R_8$.

    \item For each $r\in\bR$ let $A_r=\{(x,y)\in\bR\times\bR:y=r-x^2\}$. Let $\mathcal{A}=\{A_r:r\in\bR\}$.\begin{enumerate}
    \item Prove that $\mathcal{A}$ is a partition of $\bR\times\bR$.

    Proof: 
    Consider that to be a partition, partition $\mathcal{P}$ of A must satisfy the following conditions:
    If $x \in \mathcal{P}$ then $x \neq \varnothing$.
    If $x,y \in \mathcal{P}$ then $x=y$ or $x\cap y = \varnothing$.
    $\bigcup_{x \in P} x = A$

    Therefore, $A_r\in \mathcal{A}$ is a nonempty set because $\forall r \in\bR$ $y = r-x^2$ is a parabola with infinitely many elements. 

    Suppose $A_r,A_{r'}\in \mathcal{A} $ then either $r=r' \Rightarrow A_r = A_{r'}$  or $r\neq r'$ and the parabolas 
    $y=r-x^2$ and $y=r'-x^2$ are distinct implying $A_r \cap A_{r'} = \varnothing$ 
    
    \item Briefly describe this partition graphically. (What do the elements of $\mathcal{A}$ look like in the $x,y$-plane?)

    On the $x,y$-plane the elements of $\mathcal{A}$ look like a series of inverted parabolas with expanding tails as they increase their y intercept with intercepts being both positive and negative. At $r=0$ the parabola would pass through the origin. When r is negative the y intercept of the parabola is negative and when r is positive the y intercept of the porabola is positive. 
    
    \item Define an equivalence relation $S$ on $\bR\times\bR$ such that $(\bR\times\bR)/S=\mathcal{A}$.

    Let $(x,y)S(\tilde{x} \tilde{y})$ subject to $y+x^2 = \tilde{y} \tilde{x}^2$.
    Because $y=r-x^2 \Rightarrow r=y+x^2$ if $(x,y)$ is related to $(\tilde{x} \tilde{y})$ then $r= \tilde{y} \tilde{x}^2$
    
    \item List three different elements of $S$.

    $((0,0),(-2,-4)),((0,0),(2,-4),((1,1),(0,2))\in S$
    
    \end{enumerate}

    \end{enumerate}

}
\end{document}
