\documentclass{article}
\usepackage[utf8]{inputenc}
\usepackage{amssymb,amsmath,titling,enumitem}
\usepackage[a4paper, margin=0.75in]{geometry}

\setlength{\droptitle}{-30pt}

\title{Homework 5}
\author{MAT 108}
\date{Due 11:59pm, 5/12/2023\\ 
\vspace{0.1cm}
Submit to Gradescope}

\newcommand{\bR}{\mathbb{R}}
\newcommand{\bQ}{\mathbb{Q}}
\newcommand{\bC}{\mathbb{C}}
\newcommand{\bZ}{\mathbb{Z}}
\newcommand{\bN}{\mathbb{N}}

\begin{document}

\maketitle

{\large

\begin{enumerate}[labelindent=0pt,leftmargin=0pt]

    \setlength{\itemsep}{13pt} 

    \item Let $R$ be a relation on a set $A$. Consider the three properties: reflexive, symmetric, and transitive.\begin{enumerate}
    \item Which property is equivalent to $R\subseteq R^{-1}$? Justify your answer.
    \item Which property is equivalent to $\text{Id}_A\subseteq R$? Justify your answer.
    \item Find a definition of the last remaining property that uses only sets rather than elements (similar to how parts (a) and (b) give definitions using only sets and not elements). 
    \end{enumerate}
    

    \item Define a relation on $\bR$ that is symmetric and transitive, but not reflexive. Justify your answer.

    \item Let $R$ be the relation on $\bZ$ defined by $a R b$ iff there exists $n\in\bZ$ such that $ab=n^2$. Prove that $R$ is an equivalence relation.

    \item Recall that for each $n\in\bZ$ we define a relation $R_n$ on $\bZ$ by $a R_n b$ iff $n\,|\,(a-b)$. Find a value for $n$ such that $R_n=R_{12}\circ R_8$. Prove that your answer is correct. (Optional: Can you determine the relationship between $a$, $b$, and $n$ that guarantees $R_n=R_a\circ R_b$?)

    \item For each $r\in\bR$ let $A_r=\{(x,y)\in\bR\times\bR:y=r-x^2\}$. Let $\mathcal{A}=\{A_r:r\in\bR\}$.\begin{enumerate}
    \item Prove that $\mathcal{A}$ is a partition of $\bR\times\bR$.
    \item Briefly describe this partition graphically. (What do the elements of $\mathcal{A}$ look like in the $x,y$-plane?)
    \item Define an equivalence relation $S$ on $\bR\times\bR$ such that $(\bR\times\bR)/S=\mathcal{A}$.
    \item List three different elements of $S$.
    \end{enumerate}

    
    \end{enumerate}

}
\end{document}