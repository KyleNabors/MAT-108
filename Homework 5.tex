\documentclass{article}
\usepackage[utf8]{inputenc}
\usepackage{amssymb,amsmath,titling,enumitem}
\usepackage[a4paper, margin=0.75in]{geometry}

\setlength{\droptitle}{-30pt}

\title{Homework 5}
\author{MAT 108}
\date{Due 11:59pm, 5/12/2023\\ 
\vspace{0.1cm}
Submit to Gradescope}

\newcommand{\bR}{\mathbb{R}}
\newcommand{\bQ}{\mathbb{Q}}
\newcommand{\bC}{\mathbb{C}}
\newcommand{\bZ}{\mathbb{Z}}
\newcommand{\bN}{\mathbb{N}}

\begin{document}

\maketitle

{\large

\begin{enumerate}[labelindent=0pt,leftmargin=0pt]

    \setlength{\itemsep}{13pt} 

    \item Let $R$ be a relation on a set $A$. Consider the three properties: reflexive, symmetric, and transitive.\begin{enumerate}
    \item Which property is equivalent to $R\subseteq R^{-1}$? Justify your answer.

    The inverse relation $R^{-1}$ is defined as ${(y, x) | (x, y) \in R}$. So, if $R\subseteq R^{-1}$, then for every  $(x, y)$ in $R$, the pair $(y, x)$ must also be in $R$.

    A symmetric relationship is defined as for all $x, y \in A$, if $(x, y) \in R$ then $(y, x) \in R$. Thus is must be that $R\subseteq R^{-1}$ is equivalent to the symmetric property. 
    
    \item Which property is equivalent to $\text{Id}_A\subseteq R$? Justify your answer.

    The identity relation on $A$, denoted by $\text{Id}_A$, is defined as ${(x, x) | x \in A}$. 
    Thus, if $\text{Id}_A\subseteq R$, then every pair of the form $(x, x)$ is in $R$.

    A relation $R$ on a set $A$ is reflexive iff, for all $x \in A$, $(x, x) \in R$. 
    Thus, it must be that $\text{Id}_A\subseteq R$ is reflexive. 
    
    \item Find a definition of the last remaining property that uses only sets rather than elements (similar to how parts (a) and (b) give definitions using only sets and not elements).

    The last of the three remaining properties is transitive. 
    The following is an alternative definition of transitive. 
    A relation $R$ is transitive iff $R \circ R \subseteq R$.
    
    \end{enumerate}
    

    \item Define a relation on $\bR$ that is symmetric and transitive, but not reflexive. Justify your answer.

    One simple relation on $\mathbb{R}$ that is symmetric and transitive, but not reflexive, is the "not-equal-to" relation. We define this relation $R$ as follows: for all $x, y \in \mathbb{R}$, $(x, y) \in R$ if and only if $x \neq y$.

    Let's justify that this relation satisfies the properties:
    
    Symmetric: If $x \neq y$, then $y \neq x$. Hence, if $(x, y) \in R$, then $(y, x) \in R$. So, $R$ is symmetric.
    Transitive: If $x \neq y$ and $y \neq z$, then $x \neq z$. Hence, if $(x, y) \in R$ and $(y, z) \in R$, then $(x, z) \in R$. So, $R$ is transitive.
    Not reflexive: For any $x \in \mathbb{R}$, we have $x = x$, so $(x, x) \notin R$. Hence, $R$ is not reflexive.
    Thus, the "not-equal-to" relation on $\mathbb{R}$ satisfies the required properties.

    \item Let $R$ be the relation on $\bN$ defined by $a R b$ iff there exists $n\in\bN$ such that $ab=n^2$. Prove that $R$ is an equivalence relation.

    \item Recall that for each $n\in\bZ$ we define a relation $R_n$ on $\bZ$ by $a R_n b$ iff $n\,|\,(a-b)$. Find a value for $n$ such that $R_n=R_{12}\circ R_8$. Prove that your answer is correct. (Optional: Can you determine the relationship between $a$, $b$, and $n$ that guarantees $R_n=R_a\circ R_b$?)

    \item For each $r\in\bR$ let $A_r=\{(x,y)\in\bR\times\bR:y=r-x^2\}$. Let $\mathcal{A}=\{A_r:r\in\bR\}$.\begin{enumerate}
    \item Prove that $\mathcal{A}$ is a partition of $\bR\times\bR$.
    \item Briefly describe this partition graphically. (What do the elements of $\mathcal{A}$ look like in the $x,y$-plane?)
    \item Define an equivalence relation $S$ on $\bR\times\bR$ such that $(\bR\times\bR)/S=\mathcal{A}$.
    \item List three different elements of $S$.
    \end{enumerate}

    
    \end{enumerate}

}
\end{document}