
\documentclass{article}
\usepackage[utf8]{inputenc}
\usepackage{amssymb,amsmath,titling,enumitem}
\usepackage[a4paper, margin=0.75in]{geometry}
\setlength{\droptitle}{-30pt}
\title{Homework 2}
\author{MAT 108}
\date{Due 11:59pm, 4/14/2023\\
\vspace{0.1cm}
Submit to Gradescope}
\newcommand{\bR}{\mathbb{R}}
\newcommand{\bQ}{\mathbb{Q}}
\newcommand{\bC}{\mathbb{C}}
\newcommand{\cB}{\mathcal{B}}
\newcommand{\cC}{\mathcal{C}}
\newcommand{\bZ}{\mathbb{Z}}
\newcommand{\bN}{\mathbb{N}}
\newcommand{\cG}{\mathcal{G}}
\newcommand{\cH}{\mathcal{H}}
\newtheorem{theorem}{Theorem}
\begin{document}
\maketitle
{\large
\noindent We use the following notation:
\begin{description}[itemsep=0em]
\item $\bR$: real numbers
\item $\bC$: complex numbers, $a+b\sqrt{-1}$ for $a,b\in\bR$
\item $\bZ$: integers
\item $d\,|\,n$: $d$ divides $n$, meaning $(\exists
q\in\bZ)(n=dq)$
\end{description}
\begin{enumerate}[labelindent=0pt,leftmargin=0pt]
\setlength{\itemsep}{13pt}
\item Find negations of the following propositions that do not use the
symbol $\sim$. Then determine, with brief justification, whether the given
proposition or your negation is true.\begin{enumerate}

\item $(\exists x\in\bC)(x^2\in\bR\land x^4 < 0)$
\subitem $\sim\!(\exists x\in\bC)(x^2\in\bR\land x^4 < 0)$
\subitem $(\forall x\in\bC)\sim\!(x^2\in\bR\land x^4 < 0)$
\subitem $(\forall x\in\bC)(x^2\notin\bR\lor x^4 \geq 0)$

\item $(\exists d\in\bZ)(\forall n\in \bZ)\sim\!(d\,|\,n)$
\subitem $\sim\!(\exists d\in\bZ)(\forall n\in \bZ)\sim\!(d\,|\,n)$
\subitem $(\forall d\in\bZ)\sim\!(\forall n\in \bZ)\sim\!(d\,|\,n)$
\subitem $(\forall d\in\bZ)(\exists n\in \bZ)(d\,|\,n)$

\item $(\forall x\in\bR)(x>0\Rightarrow(\exists y\in\bR)(2^y=x))$
\subitem $\sim\!(\forall x\in\bR)(x>0\Rightarrow(\exists y\in\bR)(2^y=x))$
\subitem $(\exists x\in\bR)(x>0\Rightarrow(\exists y\in\bR)(2^y=x))$
\subitem $(\exists x\in\bR)(x\leq0\land(\forall y\in\bR)(2^y \neq x))$

\item $(\forall a\in\bR)(\forall b\in\bR)(\forall c\in\bR)(\exists
x\in\bR)(x^3+ax^2+bx+c=0)$
\subitem $\sim\!(\forall a\in\bR)(\forall b\in\bR)(\forall c\in\bR)(\exists
x\in\bR)(x^3+ax^2+bx+c=0)$
\subitem $(\exists a\in\bR)\sim\!(\forall b\in\bR)(\forall c\in\bR)(\exists
x\in\bR)(x^3+ax^2+bx+c=0)$
\subitem $(\exists a\in\bR)(\exists b\in\bR)\sim\!(\forall c\in\bR)(\exists
x\in\bR)(x^3+ax^2+bx+c=0)$
\subitem $(\exists a\in\bR)(\exists b\in\bR)(\exists c\in\bR)\sim\!(\exists
x\in\bR)(x^3+ax^2+bx+c=0)$
\subitem $(\exists a\in\bR)(\exists b\in\bR)(\exists c\in\bR)(\forall
x\in\bR)\sim\!(x^3+ax^2+bx+c=0)$
\subitem $(\exists a\in\bR)(\exists b\in\bR)(\exists c\in\bR)(\forall
x\in\bR)(x^3+ax^2+bx+c \neq 0)$

\end{enumerate}


\item Express the following proposition using only the symbols we have
defined in class (you cannot use English words or define extra symbols):
\textit{For any nonzero real number $x$, there is exactly one real number
$y$ such that $xy=1$.}

\subitem $(\forall x \in \mathbb R_{\ne 0})(\exists!y \in\bR)(xy=1)$


\item Suppose while reading a proof that you come across the following
argument: \textit{We have shown that if a square matrix is not invertible
then it has determinant $0$, and we have shown that a square matrix with
two identical rows has determinant $0$. Therefore a square matrix with two
identical rows is not invertible.} Is this reasoning correct? Justify your
answer by labeling the assertions in this statement $P$, $Q$, and $R$ and
making an appropriate truth table.

\subitem Let P be the statement a square matrix is  not invertible

\subitem Let Q be the statement a square matrix has a determinant $0$

\subitem Let R be the statement a square matrix has two identical rows

\subitem We can rewrite the given argument as:
\subitem $[(P \Rightarrow Q) \land (R \Rightarrow Q)] \Rightarrow (R \Rightarrow P)$
\subitem We can check this argument with a truth table 
\subitem Given the tables output we can see that the argument outlined in the proof is correct. 

\begin{table}[]
    \centering
    \begin{tabular}{c|c|c|c|c|c|c}
        P & Q & R & $P \Rightarrow Q$ & $R \Rightarrow Q$ & $R \Rightarrow P$ & $[(P \Rightarrow Q) \land (R \Rightarrow Q)] \Rightarrow (R \Rightarrow P)$ \\
        T & T & T & T & T & T & T \\
        T & T & F & T & T & T & T \\
        T & F & T & F & F & T & T \\
        T & F & F & F & T & T & T \\
        F & T & T & T & T & F & F \\
        F & T & F & T & T & F & F \\
        F & F & T & T & F & T & T \\
        F & F & F & T & T & T & T \\
    \end{tabular}
    \caption{Caption}
    \label{tab:my_label}
\end{table}

\item Use a direct proof to show that if $a,b\in\bZ$ are both positive
and $a\,|\,b$ and $b\,|\,a$, then $a=b$.

Proposition: If $a,b\in\bZ$ are both positive and $a\,|\,b$ and $b\,|\,a$, then $a=b$.

\begin{therom}
Proof: Let $a,b\in\bZ$ 
By definition there exists some $x\in\bZ$ such that $a*b = x$
We can rewrite this statement as $a = bx$ and $b = ax$
Therefore we can rewrite $a\,|\,b$ as  $a\,|\,ax$ and $b\,|\,a$ as $b\,|\,bx$
Given that $a*b = x$ this implies  $a\,|\,b = ab$ and $b\,|\,a =ab$ 
This implies $a\,|\,b = b\,|\,a$
By definition of cross multiplication $a^2 = b^2$
Given that both a and b are positive this implies $a = b$ 
\qedsymbol
\end{therom}

\item We say an integer $n$ is even if and only if $2\,|\,n$, and odd
otherwise. Use a proof by contraposition to show that if there exists a
positive integer $k$ such that $n^k$ is odd, then $n$ is odd.

\item Suppose you have just proved that if $n\in\bZ$ is odd then $n+1$
is even. Use this fact in a proof by cases to show that $n^2-3n+8$ is even
for all $n\in\bZ$.
\end{enumerate}
}
\end{document}
